\documentclass[a4paper,oneside]{memoir}
\usepackage[english]{babel}
\usepackage[T1]{fontenc}
\usepackage[utf8]{inputenc}
\usepackage{wallpaper}
\usepackage{palatino}
\counterwithout{section}{chapter}
% Setup captions
%\captionstyle[\centering]{\centering}
%\changecaptionwidth
%\captionwidth{0.8\linewidth}

% Protect against widows and orphans
%\clubpenalty=10000
%\widowpenalty=10000

%\linespread{1.2}

%\raggedbottom

%\chapterstyle{ger}

%\maxsecnumdepth{subsection}

%%  Setup fancy style quotation
%%  ==================================================================
%\usepackage{tikz}
%\usepackage{framed}

%\newcommand*\quotefont{\fontfamily{fxl}} % selects Libertine for quote font

% Make commands for the quotes
%\newcommand*{\openquote}{\tikz[remember picture,overlay,xshift=-15pt,yshift=-10pt]
%     \node (OQ) {\quotefont\fontsize{60}{60}\selectfont``};\kern0pt}
%\newcommand*{\closequote}{\tikz[remember picture,overlay,xshift=15pt,yshift=5pt]
%     \node (CQ) {\quotefont\fontsize{60}{60}\selectfont''};}

% select a colour for the shading
%\definecolor{shadecolor}{rgb}{1,1,1}

% wrap everything in its own environment
%\newenvironment{shadequote}% 
%{\begin{snugshade}\begin{quote}\openquote}
%{\hfill\closequote\end{quote}\end{snugshade}}

%%  Begin document
%%  ==================================================================
\begin{document}

%%  Begin title page
%%  ==================================================================
    \thispagestyle{empty}
    \ULCornerWallPaper{1}{ku-coverpage/diku-en.pdf}
    \ULCornerWallPaper{1}{ku-coverpage/diku-en.pdf}
    \begin{adjustwidth}{-3cm}{-1.5cm}
    \vspace*{-1cm}
    \textbf{\Huge Project outside course scope} \\
    \vspace*{2.5cm} \\
    \textbf{\Huge Accelerating Ocean Modelling} \\
    \vspace*{.1cm} \\
    {\huge Adressing performance bottlenecks of the ocean modelling framework Veros} \\
    \begin{tabbing}
    % adjust the hspace below for the longest author name
    Till Grenzdörffer \hspace{1cm} \= \texttt{vmt184@alumni.ku.dk} \\
    \\[12cm]
    \textbf{\Large Supervisor} \\
    Cosmin Eugen Oancea \> \texttt{cosmin.oancea@di.ku.dk} \\
    \end{tabbing}
    \end{adjustwidth}
    \newpage
    \ClearWallPaper



%%  ==================================================================
%%  End title page
\section{Introduction}
Currently, many scientists use purely sequential software for ocean modelling, leading to long simulation times andinefficient use of modern hardware.The aim of this project is to tackle this problem by introducing highly parallel code that uses the potential of modern GPUs to accelerate the modelling process.
\section{Tridiagonal Solver}
One of the bottlenecks within veros is solving many tridiagonal systems.... 
\subsection{Trivial Algorithm}
\subsection{Trivial Algorithm - Coalesced}
In the trivial algorithm, successive iterations of the recurrent loop access neighboring data in memory.
Since we assign one problem per thread, neighboring threads access data with the stride of the size of the dimension of the tridiagonal system. This means that the accesses are uncoalesced and we can improve the performance significantly by changing the underlying data layout. 
In this case the change is trivial - we just need to transpose the input matrix, which can be done efficiently using shared-memory tiling and tranpose the result back afterwards. 

The benchmarks show clearly that the benefit of the transposition on the main kernel outweighs the cost of the transposition and thus increases the performance by a factor of XXXX.
\subsection{Flat version}
The flat version of the tridiagonal solver is based on.. \cite{andreetta2016finpar} 
\subsection{Flat version in a single kernel}
\subsection{Precision}
\subsection{Futhark}
\subsection{Benchmarks}

\section{Turbulent kinetic energy}
\subsection{Components}
\subsection{Superbee scheme}

\section{Interfacing with Jax}
Uses XLA
Integrating CUDA code with Cython
Needs modified compiler... blah



\bibliographystyle{alphadin}
\bibliography{main}
\end{document}
%%  ==================================================================
%%  End document
\iffalse %project description

Currently, many scientists use purely sequential software for ocean modelling, leading to long simulation times andinefficient use of modern hardware.The aim of this project is to tackle this problem by introducing highly parallel code that uses the potential of modern GPUsto accelerate the modelling process.Specifically the student will try to enhance the performance of the library Veros using massively GPUs.In this library there exist a few specific bottlenecks that significantly slow down the modelling process. For example,since the modelling process uses a grid-like structure, there is the need to quantify the effect of water turbulence thatis a lot smaller than one grid cell on the large-scale flow of the ocean.This problem and another two of the most important bottlenecks are already implemented using different performance-focused frameworks like JAX, Numba, CuPy, etc. with the aim of finding a suitably performant solution. However, thereis hope that these implementations can be improved further or that other languages like CUDA and Futhark can yieldhigher performance gains.Within the project, the student implements these bottlenecks in the GPU-accelerated languages CUDA and Futhark,aiming to achieve a solution that performs better than the existing implementations. This can potentially save a lot of timefor the scientists running the experiments and possibly also reduce the energy consumption required for the modellingprocess by using the hardware more efficiently.This involves interfacing the new implementations with the existing codebase, more specifically with the frameworkJAX, through either passing pointers to the GPU memory between the libraries or registering XLA “custom calls” withJAX. These XLA “custom calls” would allow the direct use of C++ or CUDA code (possibly generated by Futhark) withinJAX.Additionally this project requires the use of profiling tools for evaluating the performance of the implementations as wellas potentially identifying further bottlenecks in the framework.Learning Goals:• Understanding the basic structure of the ocean modelling library Veros• Efficient algorithm design in different parallel languages (CUDA/Futhark)• Identification of performance bottlenecks in parallel programs• Interfacing Futhark and CUDA implementation with existing python codebases• Using XLA custom calls to use C++/CUDA code within JAX*29/01/2021WrittenAccelerating Ocean Modelling
\fi